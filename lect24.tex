In two different frames of references $S$ and $S^\prime$:

$$m^2c^4 = E^2-p^2c^2={E^\prime}^2-\left(p^\prime c\right)^2$$

When

$$E=\gamma_p mc^2$$
$E^\prime = \gamma^\prime mc^2$
Denote $$\gamma_p^\prime = \frac{1}{\sqrt{1-\frac{{u_p^\prime}^2}{c^2}}}$$

\subsection{Transformation of energy and momentum}

Proper time is $\Delta \tau = \Delta t \left(1-\frac{u^2}{c^2}\right)^\frac{1}{2}$. Velocity $$u_x = \frac{dx}{d\tau} = \frac{dx}{d\tau}\frac{d\tau}{dt} = \frac{dx}{d\tau} \left(\frac{dt}{d\tau}\right)^\frac{1}{2}$$.

Then $u_x = \frac{dx}{d\tau}\frac{1}{\gamma p}$ and:

$$\begin{cases}
E   = \gamma_p mc^2 = mc^2 \frac{dt}{d\tau} =& \frac{mc^2}{d\tau} dt\\
p_x = \gamma_p mu_x = m \frac{dx}{d\tau} =& \frac{mc^2}{d\tau} dx\\
p_y = m \frac{dy}{d\tau} =& \frac{mc^2}{d\tau} dy\\
p_z = m \frac{dz}{d\tau} =& \frac{mc^2}{d\tau} dz\\
\end{cases}$$



Since $m$, $\tau$ and $c$ are equal in all frames of reference, $E, \vec{p}$ are proportional to $t, \vec{p}$.

Transformation for energy and momentum is:

$$\begin{cases}
p_x^\prime = \gamma \left(p_x - \frac{v}{c^2}E\right)\\
p_y^\prime =p_y\\
p_z^\prime = p_z\\
E^\prime   = \gamma \left( tE  - vp_x \right)\\
\end{cases}$$

Note:
$$\vec{p} = \vec{u} \frac{E}{c^2}$$
and
$$pc = \sqrt{E^2-m^2c^4}$$

\paragraph{Example} Two photons with energy $E_\gamma = 0.9 MeV$ collide and produce $e^+$ and $e^-$. What is momentum of $e^-$?

First of all $\vec{p}_{e^-} + \vec{p}_{e^+}$.

Since $E_{e^-} + E_{e^+} = 2E_\gamma$ and $E_{e^+} = \sqrt{p_{e^+}^2c^4+m_e^2c^4} = \sqrt{p_{e^-}^2c^4+m_e^2c^4} = E_{e^-}$, then $E_{e^-} = 0.9MeV$.

$$p_{e^-} = \frac{\sqrt{E_{e^-}^2 - m_{e}c^4}}{c} = \sqrt{(0.9)^2-(0.5m)^2}\frac{MeV}{c} = 0.74 \frac{MeV}{c}$$
	
	\paragraph{}
	Why can't single photon produce electron and positron? If that happens, than in frame of reference of center of mass of $e^+$ and $e^-$ total momentum is 0. But photon has nonzero momentum, so momentum isn't conserver.
	
	\textbf{Note:} $\gamma \to e^+ = e^-$ happens in the vicinity of nucleus of atoms, since nucleus takes "excess" of momentum.
	
	\paragraph{Example} Two particles with parameters $m_1, E_1, p_1, v_1$ and $m_2, E_2, p_2, v_2$ coliide and produce third particle which is in rest.
	
	$$\vec{p}_1 + \vec{p_2} = \vec{p}_3$$
	$$\vec{p}_1=\vec{p}_2$$
	$$\frac{m_1\vec{v}_1}{\sqrt{1-\frac{v_1}{c}}} = -\frac{m_2\vec{v}_2}{\sqrt{1-\frac{v_2}{c}}}$$
	
	By applying Lorentz transformation for frame of reference $S^\prime$. Before collision:
	
	$$p^\prime_{1x} + p^\prime_{2x} = \gamma \left( p_{1x} + p_{2x} \right) - \frac{\gamma v}{c^2}\left(E_1+E_2\right)$$
	
	After collision:
	
	$$p^\prime_{1x} + p^\prime_{2x} =  p_{3x}^\prime = \gamma p_{3x} - \frac{\gamma v}{c^2}E_3$$
	
	We got $E_1+E_2 = E_3$.
	
	$$E_3  = m_3c^2 = E_1+E_2  = \frac{m_1c^2}{\sqrt{1-\frac{v_1^2}{c^2}}}+\frac{m_2c^2}{\sqrt{1-\frac{v_2^2}{c^2}}}$$
	
	Obviously, $m_3 > m_1+m_2$.
	
	$$m_1^2c^4 = E_1^2-p_1^2c^2$$
	$$m_2^2c^4 = E_2^2-p_2^2c^2$$
	$$m_3^2c^4 = E_3^2-p_3^2c^2$$
	
	But
	
	$$\left(m_1+m_2\right)^2c^4 \neq (E_1+E_2)^2 - (\vec{p}_1 + \vec{p}_2)^2c^2$$