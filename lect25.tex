\paragraph{Velocity of center of mass}
Sum of momentum in frame of reference of center of mass is 0:
$$p_{1cm} = -p_{2cm} $$
We want to find velocity of center of mass, so we apply transformation of Lorentz:

$$p_{1cm}  = \gamma_{cm} \left( p_1 - \frac{E_1}{c^2}v_cm \right)$$
$$p_{2cm}  = \gamma_{cm} \left( p_2 - \frac{E_2}{c^2}v_cm \right)$$


$$ \gamma_{cm} \left( p_1 - \frac{E_1}{c^2}v_cm \right) =  -\gamma_{cm} \left( p_2 - \frac{E_2}{c^2}v_cm \right) $$
$$p_1+p_2 = v_{cm}\left( \frac{E_2}{c^2} + \frac{E_1}{c^2} \right)$$

$$v_{cm} = \frac{p_1+p_2}{\frac{E_2}{c^2} + \frac{E_1}{c^2}}$$

Or more generally

$$v_{cm} = \frac{\sum_i p_i}{\sum_i \frac{E_i}{c^2}}$$

So $\frac{E_i}{c^2}$ is replacing Newtonian $m_i$.

\paragraph{DeBroglie Wavelength}

Since $p=\frac{h}{\lambda}$ we can define $\lambda = \frac{h}{p}$ for particles, which is very small.

\paragraph{Threshold energy}
$$p+p \to p+p+p+\bar{p}$$

For resulting particles velocity is very small, i.e. momentum is negligible. So

$$E_{0,cm} = 2\frac{m_pc^2}{\sqrt{1-\frac{v^2}{c^2}}}$$
$$E_cm = 4m_pc^2$$

Then minimal velocity needed for reaction for each proton is:

$$\sqrt{1-\frac{v^2}{c^2}}= \frac{1}{2}$$

I.e $v_p = \frac{\sqrt{3}}{2}c$.

Now lets look at two particles, one is in rest and second is colliding with it. We know that $v_{cm}= \frac{\sqrt{3}}{2}$ and $v_{2cm} =\frac{\sqrt{3}}{2}$. Then $v_2$ in lab frame of reference:

$$v_2  = \frac{v_cm+v_{2cm}}{1+ \frac{v_cm\cdot v_{2cm}}{c^2}} = \frac{4\sqrt{3}}{7}c$$

$$E_2 = \frac{m_pc^2}{\sqrt{1-\frac{v^2}{c^2}}}=7m_pc^2$$

Threshold kinetic energy when one of the protons is in rest is $6m_pc^2$.

Center of mass frame of reference is optimal, since after collision all the particles are in rest.

\section{General relativity}
\subsection{Equivalence principle}
It isn't possible to know if we're in free fall towards mass or weightless in space.